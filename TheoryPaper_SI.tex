\documentclass[11pt]{article}   	% use "amsart" instead of "article" for AMSLaTeX format
%\usepackage{geometry}                		% See geometry.pdf to learn the layout options. There are lots.
%\geometry{letterpaper}                   		% ... or a4paper or a5paper or ... 
%\geometry{landscape}                		% Activate for for rotated page geometry
\usepackage[parfill]{parskip}    		% Activate to begin paragraphs with an empty line rather than an indent

\usepackage{graphicx}				% Use pdf, png, jpg, or eps� with pdflatex; use eps in DVI mode
								% TeX will automatically convert eps --> pdf in pdflatex		
\usepackage{amssymb}
\usepackage[top=1.1in, bottom=1.1in, right=1.0in, left=1.0in]{geometry}
\usepackage{fancyhdr}
\usepackage{titling}
\usepackage{authblk}
%\usepackage{natbib}
\usepackage{cite}
\usepackage[labelformat = empty,position=top]{subcaption}
\usepackage[export]{adjustbox}
\usepackage{float}
\usepackage[hidelinks]{hyperref}
\hypersetup{
    linktoc=all
}
\usepackage[aboveskip=1pt,labelfont=bf,labelsep=period,singlelinecheck=off,size=small]{caption}

%\usepackage[round]{natbib}
%\bibliographystyle{plainnat}

\bibliographystyle{ieeetr}

\usepackage{color}
\newcommand*{\edit}[1]{{\bfseries\textcolor{red}{#1}}}

\newcommand{\Expo}{\textrm{Expo}}

\setlength{\droptitle}{-1in}
\title{Theory Paper: Supplementary Information}
\author{Grant Kinsler, Kerry Geiler-Samerotte, Dmitri Petrov}
\date{}

\begin{document}
\maketitle

%\tableofcontents
%\thispagestyle{empty}

%\setcounter{page}{1}
\section{Mathematical and Computational Details}
\subsection{Derivation of Score Metric}

To derive the score (eq. 1) used in the main text, we assume that fitness for mutant $ j $ at position $ \vec{x}_j $ in condition $ k $ is given by:
\begin{equation}
F_{jk}= \frac{h_k}{\sqrt{2\pi\sigma_k^2}}\exp\left( -\frac{\sum_{i=1}^D \left( x_{ij} - o_{ik} \right)^2}{2\sigma_k^2} \right)
\end{equation}
where fitness in condition $ k $ is represented by a Gaussian function centered at $ \vec{o}_k $, with height $ h_k $ and variance $ \sigma_k^2 $. Using this absolute fitness function, relative fitness from an ancestor at the location $ \vec{a} $ is given by:
\begin{equation}
 f_{jk} = \frac{F_{jk}}{F_{Ak}} - 1 = \frac{\exp\left( -\frac{\sum_{i=1}^D \left( x_{ij} - o_{ik} \right)^2}{2\sigma_k^2} \right)}{\exp\left( -\frac{\sum_{i=1}^D \left( a_{i} - o_{ik} \right)^2}{2\sigma_k^2} \right)} - 1 = \exp\left( \frac{\sum_{i=1}^D \left( a_{i} - o_{ik} \right)^2 - \sum_{i=1}^D \left( x_{ij} - o_{ik} \right)^2}{2\sigma_k^2} \right) - 1 
 \end{equation}
Note that the height of the Gaussian $ h_k $ drops out of the equation for relative fitness. To find values of the parameters (including location of the mutants, optima, and ancestor) that best fit our observed data (the measured relative fitnesses), we need to derive a score that minimizes the difference between the predicted relative fitness given by the model and the observed relative fitness. Since our estimation technique relies on repeatedly evaluating a function of the locations of the mutants, optima, and ancestor, it is computationally costly to have an exponential function in this evaluation. Accordingly, we use a log-transform of this equation to derive our score metric:
\begin{equation}
S\left(x,o,a|f,\epsilon \right) = \sum_{j=1}^M \sum_{k=1}^C \epsilon_{jk} \left[ \left( \sum_{i=1}^D \left(a_{i} - o_{ik}\right)^2 - \sum_{i=1}^D \left(x_{ij} - o_{ik}\right)^2 \right) - 2 \sigma_{k}^2 \log\left[ 1+f_{jk} \right] \right]^2
\end{equation}
where the $ \epsilon_{jk}$ is weighting for measurement error according to relative inverse variance weighting, such that:
\[ \epsilon_{jk} =  \frac{\frac{1}{s_{jk}^2}}{\sum_{j,k} \frac{1}{s_{jk}^2}}  \]
where $ s_{jk}^2 $ is the sampling variance for the given measurement [CITE venkataram supplement?]. 

\subsection{Constraining symmetry}

There are a number of symmetric solutions that can fit this solution. We constrain symmetry and reduce the number of parameters that we fit by the ancestor to the location $ \left( a, 0, \dots , 0 \right) $ and optima such that the first optimum is at the origin, and the $ d $th optimum is constrained to the $ d $-dimensional Euclidean subspace (up until the $ D $th optimum, whereafter each optimum has all $ D $ values). After imposing these constraints, and fixing the variance of the Gaussian to $ \sigma_k^2 = 1.0 $, we can calculate the total number of variables in our optimization problem to understand when the problem becomes underdetermined. We need the number of variables to be less than the number of data points (which is given by $ M C$). Thus we are overdetermined (and okay solving the problem if):
\[ M(D) + (C-D)D + \frac{D(D+1)}{2}  < M C \]
When there are many more mutants than conditions, then the terms with $ M $ dominate, and the problem is feasible when the number of dimensions is less than the number of conditions included.

%where $ \vec{x}_j $ is a vector representing the location of genotype $ j $, $ o_k $ represents the location of the optimum of environment $ k $, $ \sigma_k^2 $ is the variance of the Gaussian fitness function, and $ D $ is the number of phenotypic dimensions being evaluated. Since we measure fitness relative to the ancestor, we can translate to the relative fitness of genotype $ j $ in environment $ k $:
%
%\begin{equation} 
%%f_{jk} = \frac{W_{k}\left( \vec{x}_j \right)}{W_{k}\left( A \right)} -1 = \frac{\exp\left( -\frac{\sum_{i=1}^D \left( x_{ij} - o_{ik} \right)^2}{2\sigma^2} \right)}{\exp\left( -\frac{\sum_{i=1}^D \left( A_i- o_{ik} \right)^2}{2\sigma^2} \right)} -1 = \exp\left(\frac{\sum_{i=1}^D \left( A_i- o_{ik} \right)^2 - \sum_{i=1}^D \left( x_{ij} - o_{ik} \right)^2}{2\sigma^2}  \right) -1 
%f_{jk} = \frac{F_{k}\left( \vec{x}_j \right)}{F_{k}\left( A \right)} -1 = \exp\left(\frac{\sum_{i=1}^D \left( a_i- o_{ik} \right)^2 - \sum_{i=1}^D \left( x_{ij} - o_{ik} \right)^2}{2\sigma_k^2}  \right) -1 
% \end{equation}
%
%We want to minimize the difference between this fitness value for the model and the measured fitness value. For computational feasibility, we take the log transform of this difference, and perform least squares optimization summing over all mutants and conditions. Thus, we are interested in finding the global minimum of the function:
%
%\begin{equation}
%S\left(x,o,A|f,\epsilon \right) = \sum_{j=1}^M \sum_{k=1}^C \epsilon_{jk} \left[ \left( \sum_{i=1}^D \left(x_{ij} - o_{ik}\right)^2 - \left(  (A-o_{1k})^2 + \sum_{i=2}^D o_{ik}^2   \right) \right) - 2 \log\left[ 1+f_{jk} \right] \right]^2
%\end{equation}



%There are a number of symmetric solutions that can fit this solution. We constrain symmetry and reduce the number of parameters that we fit by the ancestor to the location $ \left( A, 0, \dots , 0 \right) $ and optima such that the first optimum is at the origin, and the $ d $th optimum is constrained to the $ d $-dimensional Euclidean subspace (up until the $ D $th optimum, whereafter each optimum has all $ D $ values). After imposing these constraints, and fixing the variance of the Gaussian to $ \sigma_k^2 = 1.0 $, we can calculate the total number of variables in our optimization problem to understand when the problem becomes underdetermined. We need the number of variables to be less than the number of data points (which is given by $ M C$). Thus we are overdetermined (and okay solving the problem if):
%\[ M(D) + (C-D)D + \frac{D(D+1)}{2}  < M C \]
%When there are many more mutants than conditions, then the terms with $ M $ dominate, and you need approximately one more condition than the number of dimensions you'd like to estimate.

\subsection{Optimization Implementation}

To find the best possible solution to this nonlinear optimization problem, we use a global minimization technique implemented in Python. In particular, we use \texttt{scipy.optimize.basinhopping}, which combines gradient descent with random displacement to find various local minima. Local minima are accepted similarly to Metropolis-Hastings and Simulated Annealing approaches, and the process continues to iterate until the number of steps has been completed. We run numerous simulations from random starting locations and compare the found solutions to ensure consistency in our optima. However, because this is a global optimization technique, there is no guarantee that the solution we find is the best possible one. Thus, our solutions should be seen as the best local optimum found. Because of this, substantial simulations are needed to confirm the robustness of this model and its implementation.

\subsection{Simulation Implementation}

To generate simulation data, we uniformly draw points from within the $ n$-dimensional ball according to the Marsaglia Algorithm \cite{Marsaglia1972}





\end{document}  